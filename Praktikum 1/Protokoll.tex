%\documentclass{proc}

\documentclass[10pt]{article}
\usepackage{lipsum}% http://ctan.org/pkg/lipsum
\usepackage{calc}% http://ctan.org/pkg/calc
\usepackage[
  letterpaper,%
  textheight={47\baselineskip+\topskip},%
  textwidth={\paperwidth-126pt},%
  footskip=75pt,%
  marginparwidth=0pt,%
  top={\topskip+0.75in}]{geometry}% http://ctan.org/pkg/geometry

\usepackage[ngerman]{babel}
\usepackage[utf8]{inputenc}
\usepackage{lastpage} % for the number of the last page in the document
\usepackage{fancyhdr}
\pagestyle{fancy}
\fancyhf{}
\lhead{Verteilte Systeme Protokoll Aufgabe 1}
\rhead{Fabian Herbert, Rolf Huesmann}
\lfoot{\today}
\rfoot{Seite \thepage\ von \pageref{LastPage}}

\begin{document}
\section{Zielsetzung Aufgabe 1 (3D Drucker)}
Wir möchten in der ersten Aufgabe einen 3D-Drucker simulieren mit einem Druckkopf und drei Patronen.
Unser Display wird eine Text-Ausgabe haben, welche Aufträge annimmt und Füllstände jeder Patrone ausgibt.
Wenn eine Patrone leer ist, muss auf dem Display die nachgefüllte Menge eingegeben werden.
\\
\\
Die Druckaufträge möchten wir als ZIP-File vorhalten. Das Zip-File soll die Datenmenge Simulieren, die zum Drucker übermittelt werden. Dort könnte pro Druck-Ebene ein Bild mit zu druckenden Pixeln gespeichert sein.\\
\\
\\
\textbf{Druckvorgang:}
\begin{itemize}
\item Pro Kilobyte soll ein Gramm aus einer Patrone verbraucht werden.
\item Nach einer Aufwärmphase von 10 Sekunden kann der Drucker alle zwei Sekunden ein Gramm Plastik verarbeiten.
\item Zwischen zwei Druckaufträgen muss der Druckkopf erst 20 Sekunden gereinigt werden.
\end{itemize}
\\
Wir planen die Aufgabe erst mit UDP zu realisieren, weil dort die Fehlerbehandlung ausgereifter sein müssen als bei TCP.
Im Anschluss tauschen wir das UDP-Modul gegen TCP aus. Dafür sollten wir dann keine extra Fehlerbehandlung entwickeln brauchen.
\\
\\
Wir werden das System mit Java aufbauen und mit Junit eine Testumgebung nutzten.

\section{Protokoll}
Wir haben wie geplant den Druckerkopf, Toner und das Pannel erst mit UDP realisiert. 
Dann im Anschluss die UDP Module mit TCP Modulen ersetzt.
Wir können nun durch setzten einer Bool Variable zwischen den beiden Modulen wechseln. 

\subsection{Server/Client rollen}
Dabei ist der Toner ein Server, welcher Anfragen vom Druckerkopf beantwortet. Der Druckerkopf ist hingegen zunächst ein Server der auf Anfragen vom Pannel reagiert. Nach dem übertragen der zu druckenden Datei wechseln Druckerkopf und Pannel die Rollen Der Druckerkopf wird zum Client um dem Pannel den Status des Druckvorgangs zu übertragen.

\subsection{UDP vs. TCP}
\subsubsection{Testaufbau}
Als Test drucken wir fünf mal die Selbe Datei und messen die benötigte Zeit. Dies wiederholen wir auf verschieden Hardware.
\subsubsection{Ergebnisse}
\begin{tabular}{ccc}
 UDP 	 & TCP	  & Dateigröße&\\
 1m 6s   & 1m 5s  & 2,83 kb \\
 3m 24s  & 3m 20s & 16,7mb \\
 12m 16s & 13m 0s & 86,8 mb \\
\end{tabular}
\subsubsection{Fazit}
Bei unseren Tests hat sich erstaunlicherweise heraus gestellt, dass TCP schneller ist als UDP. Wir haben erwartet, das UDP bei kleinen Daten schneller ist als TCP. Und es irgendwo ein Punkt gibt an dem TCP UDP überholt. \\
\\
Weil UDP nach unseren Erfahrungen 'pflegeleichter' beim Programmierern ist, entscheiden wir uns für UDP.
Im weiterem Verlauf des Praktikums können wir uns durch ändern einer einzigen Variable immer noch spontan um entscheiden.

\section{Zielsetzung Aufgabe 2 (Dashbord)}
Um mehrere Drucker verwalten zu können entwickeln wir ein Dashbord. Jeder Drucker bekommt eine ID zu gewissen und ist über diese ansprechbar.
Das Dashbord hat immer den Überblick über den Status der Drucker, deren Aufträge Bezeichnungsweise den Druckwarteschlangen.
\\
Über das Rest kommuniziert das Dashbord mit den Druckern.

%
%
%\section{Zielsetzung Aufgabe 3 (MQTT-Broker)}
%Der MQTT-Broker leitet Aufträge von der Smartphone-Applikation weiter an den ausgewählten Drucker. \\
%Des weiteren schickt er die Druckerstatusmeldungen an das Dashbord, welches dadurch immer den Überblick behält.\\
%Es ermittelt die Druckkosten und gibt diese weiter an das Smartphone.\\

\subsection{Ziel zum nächstem Praktikum}
Zum nächstem Termin möchten wir dem Drucker über REST ansteuern können.
\end{document}
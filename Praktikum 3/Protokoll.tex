%\documentclass{proc}

\documentclass[10pt]{article}
\usepackage{lipsum}% http://ctan.org/pkg/lipsum
\usepackage{calc}% http://ctan.org/pkg/calc
\usepackage[
  letterpaper,%
  textheight={47\baselineskip+\topskip},%
  textwidth={\paperwidth-126pt},%
  footskip=75pt,%
  marginparwidth=0pt,%
  top={\topskip+0.75in}]{geometry}% http://ctan.org/pkg/geometry

\usepackage[ngerman]{babel}
\usepackage[utf8]{inputenc}
\usepackage{lastpage} % for the number of the last page in the document
\usepackage{fancyhdr}
\pagestyle{fancy}
\fancyhf{}
\lhead{Verteilte Systeme Protokoll Aufgabe 3}
\rhead{Fabian Herbert, Rolf Huesmann}
\lfoot{\today}
\rfoot{Seite \thepage\ von \pageref{LastPage}}

\begin{document}
\section{Zielsetzung Aufgabe 3 (MQTT-Broker)}
Der MQTT-Broker leitet Aufträge von der Smartphone-Applikation weiter an den ausgewählten Drucker. \\
Des weiteren schickt er die Druckerstatusmeldungen an das Dashbord, welches dadurch immer den Überblick behält.\\
Es ermittelt die Druckkosten und gibt diese weiter an das Smartphone.\\

\section{Protokoll}
Wir haben unser Ziel mit zwei MQs umgesetzt. Eine Queue in der alle Druck- und Statusaufträge von dem Dashboard und den Cliends an die Drucker gerichtet werden. Und eine Queue in der die Drucker ihren Status schicken. Das Dashboard hört und sendet in beiden Queues. 

Die Drucker melden sich nach dem Hochfahren an beiden Queues an und senden einen Statusbericht damit alle Anderen ihre interne Druckerverwaltung auf den neusten Stand bringen können. Andersherum können neue Clients oder Dashboards die Drucker auffordern ihren aktuellen Status zu übermitteln.

Wenn ein Druckauftrag an einen Drucker gesendet wird bekommt das Dashboard dieses mit. Das Dashboard ermittelt die Druckerkosten, bucht die auf das Kundenkonto und teilt den noch zu zahlenden Betrag an den entsprechenden Client mit.
Die Drucker verschicken nach jeder Statusänderung ihren aktuellen Status aller Druckaufträge an Dashboard und Client.
Unsere komplette Kommunikation übertragen wir im Json-Format. 

\subsection{Server/Client rollen}
Es gibt einen MQTT-Server auf den sich alle Drucker, Clients und Dashboards verbinden und mit einander kommunizieren.

\subsection{Vorteile gegenüber REST}
Ein eleganter Vorteil gegenüber REST ist das den Kommunikationsteilnehmern egal sein kann ob und wie die Nachrichten übermittelt werden. Die Kommunikation ist sehr offen. Die Teilnehmer müssen nur dem gleichem Topic zuhören und kriegen von dem Broker alle Nachrichten übermittelt. 
Es gibt nur den einen Server der die Queue zu Verfügung stellt und nicht mehr für jeden Drucker einen eigenen REST-Server.
Bei REST kann Architektur bedingt immer nur der Client entscheiden wann er was kommunizieren will.

\subsection{Test}
Als Test machen wir ein Belastungstest. Dabei haben wir in 4 Virtuellen Maschinen jeweils ein Drucker mit einem Mobilien Client gestartet. Auf einer Virtuellen Maschine haben wir auch ein Dashboard laufen lassen. Die Drucker haben wir zufällig mit Druckaufträgen von jeweils auf anderen Virtuellen Maschine laufenden mobilen Client bedient. Wir haben insgesamt pro Client 5 Druckaufträge in kurzer Zeit übermittelt. Die Druckaufträge wurden ohne Probleme abgearbeitet und dem Client die richtigen Preise übermittelt.



\end{document}
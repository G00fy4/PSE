%\documentclass{proc}

\documentclass[10pt]{article}
\usepackage{lipsum}% http://ctan.org/pkg/lipsum
\usepackage{calc}% http://ctan.org/pkg/calc
\usepackage[
  letterpaper,%
  textheight={47\baselineskip+\topskip},%
  textwidth={\paperwidth-126pt},%
  footskip=75pt,%
  marginparwidth=0pt,%
  top={\topskip+0.75in}]{geometry}% http://ctan.org/pkg/geometry

\usepackage[ngerman]{babel}
\usepackage[utf8]{inputenc}
\usepackage{lastpage} % for the number of the last page in the document
\usepackage{fancyhdr}
\pagestyle{fancy}
\fancyhf{}
\lhead{Verteilte Systeme Protokoll Aufgabe 2}
\rhead{Fabian Herbert, Rolf Huesmann}
\lfoot{\today}
\rfoot{Seite \thepage\ von \pageref{LastPage}}

\begin{document}
\section{Zielsetzung Aufgabe 2 (Dashbord)}
Um mehrere Drucker verwalten zu können entwickeln wir ein Dashbord. Jeder Drucker bekommt eine ID zu gewissen und ist über diese ansprechbar.
Das Dashbord hat immer den Überblick über den Status der Drucker, deren Aufträge Bezeichnungsweise den Druckwarteschlangen.
\\
Über das Rest kommuniziert das Dashbord mit den Druckern.


\section{Protokoll}
Wir haben das Dashboard über REST mit verschiedenen Druckern kommunizieren lassen. Als ID verwenden wir die Server-Adresse bzw. URL. Über das Dashboard können Druckaufträge verwaltet und den verschiedenen Drucker zugeteilt werden.
Es können detaillierte Statusabfragen über einzelne Druckaufträge aufgerufen werden.

\subsection{Server/Client rollen}
Die Drucker stellen jeweils einen REST-Server zur Verfügung. Das Dashboard fragt diese regelmäßig ab um deren Status zu erfahren. Druckaufträge werden über GET-Parameter in der URL an die Drucker übermittelt.
%so viele druckaufträge wie möglich in einer minute drucken lassen als test 1minute 10 sek für einen druckazftrag
\subsection{Test}
Als Test haben wir unsere Drucker mit Druckaufträgen geflutet. Dabei haben wir gemessen, dass die Drucker für unsere Testdatei 70 Sekunden brauchen um diese zu drucken.

\section{Zielsetzung Aufgabe 3 (MQTT-Broker)}
Der MQTT-Broker leitet Aufträge von der Smartphone-Applikation weiter an den ausgewählten Drucker. \\
Des weiteren schickt er die Druckerstatusmeldungen an das Dashbord, welches dadurch immer den Überblick behält.\\
Es ermittelt die Druckkosten und gibt diese weiter an das Smartphone.\\


\subsection{Ziel zum nächstem Praktikum}
Zum nächstem Termin möchten wir dem Drucker über MQTT kommunizieren.
\end{document}